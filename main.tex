% Document class and basic setup
\documentclass[10pt]{exam}
% \usepackage[utf8]{inputenc}
% \usepackage[T1]{fontenc}

% Page layout and geometry
\usepackage[margin=0.25in, includefoot, includehead]{geometry}
\usepackage{microtype}

% Font packages - XeLaTeX/LuaLaTeX vs pdfLaTeX compatibility
\usepackage{ifxetex,ifluatex}
\ifxetex
\usepackage{fontspec}
\setmainfont{Linux Libertine O}[
  Ligatures=TeX,
  Numbers=OldStyle
]
\setsansfont{Linux Biolinum O}[
  Ligatures=TeX,
  Numbers=OldStyle
]
% \setmonofont{Latin Modern Mono}[Scale=MatchLowercase]
\setmonofont{DejaVu Sans Mono}[Scale=MatchLowercase]
\else\ifluatex
\usepackage{fontspec}
\setmainfont{Linux Libertine O}[
  Ligatures=TeX,
  Numbers=OldStyle
]
\setsansfont{Linux Biolinum O}[
  Ligatures=TeX,
  Numbers=OldStyle
]
\setmonofont{Latin Modern Mono}[Scale=MatchLowercase]
\else
\usepackage{libertine}
\fi\fi

% Math packages
\usepackage{amsmath,amssymb,amsthm}
\usepackage{mathtools}
\usepackage{bm}
\usepackage{accents}

% List and enumeration
\usepackage[shortlabels]{enumitem}
\usepackage{multicol}

% Graphics and figures
\usepackage{graphicx}
\usepackage{tikz}
\usepackage{pgfplots}
\pgfplotsset{compat=1.18}
\usetikzlibrary{shapes}
\usepackage{pgfplots}
\pgfplotsset{compat=1.18}
% \usepackage{svg}
\usepackage{graphicx}
\usepackage{float}
\usepackage{wrapfig}

% Tables and arrays
\usepackage{booktabs}
\usepackage{multirow}
\usepackage{nicematrix}

% Color and highlighting
\usepackage{xcolor}
\definecolor{answerboxcolor}{RGB}{255,245,245} % Very light red background
\usepackage{empheq}

% Captions and references
\usepackage[hypcap=false,font=small,labelfont=bf,tableposition=top]{caption}
\usepackage{subcaption}
\usepackage[colorlinks=true, linkcolor=blue, urlcolor=blue]{hyperref}
\usepackage{cleveref}

% Units and chemistry
\usepackage{siunitx}
\sisetup{
  group-digits=integer,
  group-minimum-digits=3,
  group-separator={,}
}
\DeclareSIUnit\angstrom{\text{\AA}} % \angstrom is depracted, so define it here.
\usepackage[version=4]{mhchem}
\usepackage{chemformula}

% Code listings
\usepackage{minted}
\usepackage{listings}
\AtBeginEnvironment{minted}{
\fontsize{8}{10}\selectfont}

% Utility packages
\usepackage{cancel}
\usepackage{etoolbox}
\usepackage{pdfpages}
\usepackage{blindtext}
\usepackage{lipsum}

% Header and footer setup using exam class commands
% The exam class provides its own header/footer system
\lhead{\textbf{MSEN 640-600}}
\chead{\textbf{Homework \#1}}
\rhead{\textbf{Nathaniel Thomas}}
\lfoot{}
\cfoot{\thepage}
\rfoot{}

% Add header rule
\headrule

% Custom commands
\newcommand{\fahrenheit}{^\circ{F}}
\newcommand*\widefbox[1]{\fbox{\hspace{2em}#1\hspace{2em}}}
\newcommand{\msout}[1]{\text{\sout{$#1$}}}

% SI Units
\DeclareSIUnit\year{y}

% Professional boxed answer environment
% Creates uniform, centered answer boxes with light red background
\newcommand{\boxedanswer}[1]{
  \begin{center}
    \fcolorbox{black}{answerboxcolor}{%
      \begin{minipage}{0.85\textwidth}
        \vspace{0.5em}
        #1
        \vspace{0.5em}
      \end{minipage}%
    }
  \end{center}
  \vspace{0.5em}
}

% Alternative boxed answer for nested environments (like enumerate)
\newcommand{\boxedanswersmall}[1]{
  \begin{center}
    \fcolorbox{black}{answerboxcolor}{%
      \begin{minipage}{0.75\textwidth}
        \vspace{0.3em}
        #1
        \vspace{0.3em}
      \end{minipage}%
    }
  \end{center}
  \vspace{0.3em}
}


% Additional SI unit for Fahrenheit
\DeclareSIUnit\fahrenheit{\degree F}

\begin{document}

% Include title page
\begin{titlepage}
    \begin{center}
        \vspace*{1cm}
            
        \Huge
        \textbf{Homework \#1}
            
        \vspace{0.5cm}
        \LARGE
        MSEN 640 - 600
            
        \vspace{1.5cm}
        
        By: \\

        \textbf{Nathaniel Thomas}

        On Behalf of:

        \textbf{Dr. Michael Dimistriev}

        \vfill

        \vspace{0.8cm}
            
        \includesvg[width=0.4\textwidth]{"./assets/a&m_logo.svg"}
            
        \Large
        Nuclear Engineering\\
        Texas A\&M University\\
        September 12\textsuperscript{th}, 2025
            
    \end{center}
\end{titlepage}


\pagebreak

\section*{Problem 2.4.1}

Which of the following differential equations is linear, in the sense
that, if some function $\psi(z)$ is a
solution (and this may well be a different function for each equation
below), so also is the function
$\phi(z) = a\psi(z)$, where $a$ is an arbitrary constant? Justify your answers.

\begin{enumerate}[(i)]
  \item $z\frac{d\psi(z)}{dz} + g(z)\psi(z) = 0$ where $g(z)$ is some
    specific function

    \boxedanswersmall{
      Some answer:
      \begin{align*}
        1 + 1 = 2
      \end{align*}
    }

  \item $\psi(z)\frac{d\psi(z)}{dz} + \psi(z) = 0$

    \boxedanswersmall{
      Some answer:
      \begin{align*}
        1 + 1 = 2
      \end{align*}
    }

\end{enumerate}

\pagebreak

\section*{Problem 2.6.1}

An electron is in a potential well of thickness \SI{1}{\nano\meter},
with infinitely high potential barriers on either
side. It is in the lowest possible energy state in this well. What
would be the probability of finding the
electron between \SI{0.1}{\nano\meter} and \SI{0.2}{\nano\meter} from
one side of the well?

\boxedanswer{
  Some answer:
  \begin{align*}
    1 + 1 = 2
  \end{align*}
}

\pagebreak

\section*{Problem 2.6.2}

Which of the following functions have a definite parity relative to
the point $x = 0$ (i.e., we are
interested in their symmetry relative to $x = 0$)? For those that
have a definite parity, state whether it
is even or odd.

\begin{enumerate}[(i)]
  \item $\sin(x)$

    \boxedanswersmall{
      Some answer:
      \begin{align*}
        1 + 1 = 2
      \end{align*}
    }

  \item $\exp(ix)$

    \boxedanswersmall{
      Some answer:
      \begin{align*}
        1 + 1 = 2
      \end{align*}
    }

\end{enumerate}

\pagebreak

\section*{Problem 2.6.3}

Consider the problem of an electron in a one-dimensional "infinite"
potential well of width $L_z$ in
the $z$ direction (i.e., the potential energy is infinite for $z < 0$ and
  for $z > L_z$, and, for simplicity, zero
for other values of $z$). For each of the following functions, in
exactly the form stated, is this function
a solution of the time-independent Schrödinger equation?

\begin{enumerate}[(i)]
  \item $\sin(7\pi z / L_z)$

    \boxedanswersmall{
      Some answer:
      \begin{align*}
        1 + 1 = 2
      \end{align*}
    }

  \item $\cos(2\pi z / L_z)$

    \boxedanswersmall{
      Some answer:
      \begin{align*}
        1 + 1 = 2
      \end{align*}
    }

\end{enumerate}

\pagebreak

\section*{Problem 2.7.1}

Which of the following pairs of functions are orthogonal on the
interval $-1$ to $+1$?

\begin{enumerate}
  \item[(i)] $x, x^2$

    \boxedanswersmall{
      Some answer:
      \begin{align*}
        1 + 1 = 2
      \end{align*}
    }

  \item[(v)] $\exp(-2\pi ix), \exp(2\pi ix)$

    \boxedanswersmall{
      Some answer:
      \begin{align*}
        1 + 1 = 2
      \end{align*}
    }

\end{enumerate}

\pagebreak

\section*{Problem 2.8.2}

An electron wave of energy \SI{0.5}{\electronvolt} is incident on an
infinitely thick potential barrier of height
\SI{1}{\electronvolt}. Is the electron more likely to be found (a) within the
first \SI{1}{\angstrom} of the barrier, or (b)
somewhere further into the barrier?

\boxedanswer{
  Some answer:
  \begin{align*}
    1 + 1 = 2
  \end{align*}
}

\end{document}
