\documentclass[10pt]{exam}
\usepackage{amsthm}
\usepackage{libertine}
\usepackage[utf8]{inputenc}
\usepackage[T1]{fontenc}
\usepackage[margin=0.25in, includefoot]{geometry}
\usepackage{amsmath,amssymb}
\usepackage{multicol}
\usepackage[shortlabels]{enumitem}
\usepackage{cancel}
\usepackage{listings}
\usepackage{tikz}
\usepackage{float}
\usepackage{mathtools}
\usepackage{wrapfig, lipsum}
\usepackage{chemformula}
\usepackage{empheq}
\usepackage[hypcap=false]{caption}
\usepackage{subcaption}
\usepackage[colorlinks=true, linkcolor=blue, urlcolor=blue]{hyperref}
% \usepackage[normalem]{ulem}
\usepackage{graphicx}
\usepackage{bm}
\usepackage{accents}
\usepackage[T1]{fontenc}
\usepackage[font=small,labelfont=bf,tableposition=top]{caption}
\usepackage{siunitx}
\usepackage{minted}
\usepackage{etoolbox}
\usepackage{multirow}
\usepackage[version=4]{mhchem}
\usepackage{pgfplots}
\usepackage{svg}
\usepackage{microtype}
\usepackage{blindtext}
\usepackage{booktabs}
\usepackage{booktabs}
\usepackage{pdfpages}
\usepackage{nicematrix}

\pgfplotsset{compat=1.18}


\AtBeginEnvironment{minted}{
\fontsize{8}{10}\selectfont}
\newcommand{\fahrenheit}{^\circ{F}}
\newcommand*\widefbox[1]{\fbox{\hspace{2em}#1\hspace{2em}}}
\newcommand{\msout}[1]{
    \text{\sout{$#1$}}
}
\usetikzlibrary{shapes}
\DeclareSIUnit\year{y}
\sisetup{
group-digits=integer,
group-minimum-digits=3,
group-separator={,}
}
% \newcommand{\class}{NUEN 604 - 600}
% \newcommand{\examnum}{Homework 1}
% \newcommand{\examdate}{September 11\textsuperscript{th}}
% \newcommand{\timelimit}{}
% \newcommand{\laplace}{\mathcal{L}}

\newcommand{\boxedanswer}[3][0.93]{
    \fbox{
        \begin{minipage}{#1\textwidth}
            #2
        \end{minipage}
        }
    }


% Additional SI unit for Fahrenheit
\DeclareSIUnit\fahrenheit{\degree F}

\begin{document}

% Include title page
% Title page with no headers/footers
\thispagestyle{empty}

\begin{titlepage}
  \begin{center}
    \vspace*{2cm}

    % Main title
    \Huge
    \textbf{Homework \#0}

    \vspace{0.8cm}

    % Course information
    \LARGE
    MSEN 604-500

    \vspace{2cm}

    % Author information
    \Large
    \textbf{Prepared by:}\\[0.5cm]
    \huge
    \textbf{Nathaniel Thomas}

    \vspace{1cm}

    \Large
    \textbf{Prepared for:}\\[0.5cm]
    \large
    Dr. Michael Dimitriyev

    \vfill

    % University logo
    \includegraphics[width=0.4\textwidth]{"./assets/a&m_logo.pdf"}

    \vspace{1cm}

    % University and date information
    \Large
    Materials Science \& Engineering\\
    Texas A\&M University\\
    \vspace{0.5cm}
    \large
    September 2, 2025

  \end{center}
\end{titlepage}


\pagebreak

\begin{enumerate}
  \item {[10 pts]} Determine if each of the following
    differentials is exact:

    \begin{enumerate}[(a)]
      \item $dz = ydx - xdy$
      \item $dz = -\frac{xy}{w^2}dw + \frac{y}{w} dx + \frac{x}{w}dy$
    \end{enumerate}
  \item {[15 pts]} Consider the following equation:
    \begin{equation*}
      x(t, u, v) = \frac{t^3 u^2}{1 - v}
    \end{equation*}
    \begin{enumerate}[(a)]
      \item If $m$ is conjugate to $t$, $n$ is conjugate to $u, p$ is
        conjugate to $v$, determine the coefficient relations.
      \item Calculate $\psi(m, u, p) = z - mt - pv$: the Legendre
        transform of $z$, exchanging $t$ and $v$ for $m$ and $p$.
        Make sure that $t$ and $v$ do not appear in $\psi(m, u, p)$.
    \end{enumerate}
  \item {[25 pts]} A gas was determine to have an internal energy of
    the form $U = U_0 + 5PV$, where $U_0$ is a constant. Given the
    three states
    \begin{itemize}[\textbullet]
      \item state A: $P = \SI{0.2}{\mega\pascal},  V =
        \SI{0.01}{\meter\cubed}$
      \item state B: $P = \SI{0.2}{\mega\pascal}, V = \SI{0.03}{\meter\cubed}$
      \item state C: $P = \SI{0.5}{\mega\pascal}, V = \SI{0.01}{\meter\cubed}$
    \end{itemize}
    \begin{enumerate}[(a)]
      \item Calculate $Q$ and $W$ for the process taking state A to
        state B along a straight path
      \item Calcualte $Q$ and $W$ for the process taking state B to
        state C along a straight path
      \item Calculate $Q$ and $W$ for the process taking state C to
        state A along a straight path
      \item Calculate $Q$ and $W$ for the process taking state A to
        state B along the parabolic path $P = P_0 +
        \kappa\left(\frac{V}{V_0} - 1\right)^2$ where
        $P_0 = \SI{0.01}{\mega\pascal}, V_0 = \SI{0.02}{\meter}$, and
        $\kappa = \SI{0.04}{\mega\pascal}$
    \end{enumerate}
  \item  {[35 pts]} Consider an isolated system (no heat, matter, or
    work may cross the system boundary)
    consisting of three internal compartments A, B, and C.
    Compartment A is 1 liter, B is 2 liters, and
    C is 4 liters. The compartments are separated by partitions; each
    partition has a valve (V1 and V2)
    which may be opened remotely. Initially the central volume (B) is
    filled with an ideal gas at 298K and
    \SI{354.6}{\kilo\pascal} and the other two (A and C) are
    completely evacuated (i.e., \SI{0}{\pascal}).

    \begin{figure}[h]
      \centering
      \includegraphics[width=0.6\textwidth]{./assets/q_4_fig.png}
    \end{figure}

    \begin{enumerate}[(a)]
      \item Consider the following processes:

        \textit{First process: The valve to the A side is opened, the gas
          expands freely into compartment A, and
          the system comes to equilibrium. Then the valve to the C side
          is opened, and the system comes
        to equilibrium again.}

        \hspace{0.5 cm}

        \textit{Second process: Both valves are opened simultaneously, the
          gas expands freely into both compart-
        ments, and the A/B/C system comes to equilibrium.}

        \hspace{0.5 cm}

        Which of the processes above produces more entropy? You must
        explain your answer completely.

      \item  The internal energy of an ideal gas is determined solely
        by its temperature. If that’s the case,
        what is the final temperature of the system in the first process?

      \item It is found that the change in entropy of the first
        process is \SI{5.5}{\joule\per\kelvin}. What can you say about
        the reversibility of the process?
    \end{enumerate}

  \item {[15 pts]}
    \begin{enumerate}[(a)]
      \item If $\psi(m) = z − mx$ is the Legendre
        transform of $z(x)$ (where $m$ is conjugate to $x$), prove
        that
        \[\frac{d^2\psi}{dm^2} = -\left[\frac{d^2z}{dx^2} \right]^{-1}\]

        Hint: the chain rule will come in handy,
        as will the identity
        $dx/dz = 1/(dz/dx)$.

      \item Describe what this implies for the graph of $\psi$
        relative to the graph of $z$.
    \end{enumerate}

\end{enumerate}

\end{document}
