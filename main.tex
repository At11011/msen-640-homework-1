\documentclass[10pt]{exam}
\usepackage{amsthm}
\usepackage{libertine}
\usepackage[utf8]{inputenc}
\usepackage[T1]{fontenc}
\usepackage[margin=0.25in, includefoot]{geometry}
\usepackage{amsmath,amssymb}
\usepackage{multicol}
\usepackage[shortlabels]{enumitem}
\usepackage{cancel}
\usepackage{listings}
\usepackage{tikz}
\usepackage{float}
\usepackage{mathtools}
\usepackage{wrapfig, lipsum}
\usepackage{chemformula}
\usepackage{empheq}
\usepackage[hypcap=false]{caption}
\usepackage{subcaption}
\usepackage[colorlinks=true, linkcolor=blue, urlcolor=blue]{hyperref}
% \usepackage[normalem]{ulem}
\usepackage{graphicx}
\usepackage{bm}
\usepackage{accents}
\usepackage[T1]{fontenc}
\usepackage[font=small,labelfont=bf,tableposition=top]{caption}
\usepackage{siunitx}
\usepackage{minted}
\usepackage{etoolbox}
\usepackage{multirow}
\usepackage[version=4]{mhchem}
\usepackage{pgfplots}
\usepackage{svg}
\usepackage{microtype}
\usepackage{blindtext}
\usepackage{booktabs}
\usepackage{booktabs}
\usepackage{pdfpages}
\usepackage{nicematrix}

\pgfplotsset{compat=1.18}


\AtBeginEnvironment{minted}{
\fontsize{8}{10}\selectfont}
\newcommand{\fahrenheit}{^\circ{F}}
\newcommand*\widefbox[1]{\fbox{\hspace{2em}#1\hspace{2em}}}
\newcommand{\msout}[1]{
    \text{\sout{$#1$}}
}
\usetikzlibrary{shapes}
\DeclareSIUnit\year{y}
\sisetup{
group-digits=integer,
group-minimum-digits=3,
group-separator={,}
}
% \newcommand{\class}{NUEN 604 - 600}
% \newcommand{\examnum}{Homework 1}
% \newcommand{\examdate}{September 11\textsuperscript{th}}
% \newcommand{\timelimit}{}
% \newcommand{\laplace}{\mathcal{L}}

\newcommand{\boxedanswer}[3][0.93]{
    \fbox{
        \begin{minipage}{#1\textwidth}
            #2
        \end{minipage}
        }
    }


% Additional SI unit for Fahrenheit
\DeclareSIUnit\fahrenheit{\degree F}

\begin{document}

% Include title page
% Title page with no headers/footers
\thispagestyle{empty}

\begin{titlepage}
  \begin{center}
    \vspace*{2cm}

    % Main title
    \Huge
    \textbf{Homework \#0}

    \vspace{0.8cm}

    % Course information
    \LARGE
    MSEN 604-500

    \vspace{2cm}

    % Author information
    \Large
    \textbf{Prepared by:}\\[0.5cm]
    \huge
    \textbf{Nathaniel Thomas}

    \vspace{1cm}

    \Large
    \textbf{Prepared for:}\\[0.5cm]
    \large
    Dr. Michael Dimitriyev

    \vfill

    % University logo
    \includegraphics[width=0.4\textwidth]{"./assets/a&m_logo.pdf"}

    \vspace{1cm}

    % University and date information
    \Large
    Materials Science \& Engineering\\
    Texas A\&M University\\
    \vspace{0.5cm}
    \large
    September 2, 2025

  \end{center}
\end{titlepage}


\pagebreak

\section*{Problem 1}

What name (optionally, pronouns) would you prefer to be called by in this class?

\boxedanswer{Nathaniel Thomas. I go by Nathaniel.}

\section*{Problem 2}

What sort of materials systems are you most interested in?

\boxedanswer{I am interested in solid mechanics. Specifically, nuclear materials
in extreme environments (corrosive, high radiation environments).}

\section*{Problem 3}

What is your undergraduate degree in?

\boxedanswer{
  Chemical Engineering
}

\pagebreak

\section*{Problem 4}

Do you have any programming experience? If so, which language(s)?

\boxedanswer{
  Yes. I am most proficient in:

  \begin{enumerate}
    \item C++
    \item C
    \item Rust
    \item Julia
    \item Python
    \item MATLAB
    \item LaTeX
  \end{enumerate}

  With some experience in:

  \begin{enumerate}
    \item Javascript
    \item FORTRAN
    \item Dart
    \item Java
    \item C\#
    \item x86 assembly
    \item WGSL
    \item Visual Basic
    \item QuickBASIC
  \end{enumerate}

}

\pagebreak
\section*{Problem 5}

Evaluate the following derivatives:
\begin{enumerate}[a.]
  \item $\frac{d\Pi}{d\rho}$ where $\Pi(\rho) = \sum_{n=0}^\infty  A_n
    \rho^n; \{ A_n \}$ are real constants (these are related to
    "virial coefficients")

    \boxedanswer{
      Apply the power rule to the formula for $\Pi(\rho)$:

      \begin{align*}
        \frac{d\Pi}{d\rho} &= \sum_{n=1}^\infty A_n\cdot n \rho^{n-1}
      \end{align*}

      Note that the $n=0$ term disappears because it is constant.
    }

  \item $\frac{df}{d\phi}$ where $f(\phi) = \phi \ln\phi + (1 - \phi)
    \ln (1 - \phi) + \chi \phi(1 - \phi)$

    \boxedanswer{
      \begin{align*}
        f(\phi) &= \overbrace{\phi \ln\phi}^{A} +
        \overbrace{(1 - \phi)
        \ln (1 - \phi)}^{B} + \overbrace{\chi \phi(1 - \phi)}^{C} \\
        \frac{df}{d\phi} &= \overbrace{(1)(\ln\phi) +
        (\phi)\left(\frac{1}{\phi}\right)}^{dA/d\phi} \\
        &  \overbrace{(-1)\ln(1 - \phi) + (1 - \phi)\left(\frac{-1}{1 -
        \phi}\right)}^{dB/d\phi} \\
        & \overbrace{\chi\left[(1)(1 - \phi) + (\phi)(-1)\right]}^{dC/d\phi} \\
        \Aboxed{\frac{df}{d\phi} &= \ln\left(\frac{\phi}{1 - \phi}\right) +
        \chi(1 - 2\phi)}
      \end{align*}
    }
  \item $\frac{d^2 f}{d\phi^2}$ where $f(\phi) = \phi\ln\phi + (1 -
    \phi)\ln(1 - \phi) + \chi\phi(1 - \phi)$

    \boxedanswer{
      \label{first_deriv}

      Differentiate the answer from part \ref{first_deriv}

      \begin{align*}
        \frac{d^2 f}{d\phi^2} &= \frac{d}{d\phi}\left[
          \ln\left(\frac{\phi}{1 - \phi}\right) +
        \chi(1 - 2\phi)\right] \\
        \frac{d^2 f}{d\phi^2} &= \frac{1 - \phi}{\phi}
        \frac{d}{dx}\left[\frac{x}{1 - x}\right] \\
        \frac{d^2 f}{d\phi^2} &= \frac{1 - \phi}{\phi}
        \left[\frac{(1)(1 - \phi) - (\phi)(-1)}{(1 - \phi)^2}\right] \\
        \frac{d^2 f}{d\phi^2} &= \frac{1 - \phi}{\phi}
        \left[\frac{1}{(1 - \phi)^2}\right] \\
        \Aboxed{\frac{d^2 f}{d\phi^2} &= \frac{1 - \phi}{\phi(1 - \phi)^2} }
      \end{align*}
    }

  \item $\left(\frac{\delta f}{\delta x}\right)_y$ where $f(x,y) = x^2
    + y^2$ and $y(x) = 1/x$

    \boxedanswer{
      \begin{align*}
        \left(\frac{\delta f}{\delta x}\right)_y &= x^2 + y^2 \\
        y(x) = 1/x \\
        \left(\frac{\delta f}{\delta x}\right)_y &= \frac{d}{dx} x^2
        + \cancel{\frac{d}{dx} y^2} \\
        \Aboxed{\left(\frac{\delta f}{\delta x}\right)_y &= 2x}
      \end{align*}

      This question somewhat confuses me. The notation
      $\left(\frac{\delta f}{\delta x}\right)_y$ means "take the
      partial derivative, holding $y$ constant." Presumably, this
      means I should ignore the fact that $y(x) = x.$ Otherwise, I
      would just substitute $y$ and the asnwers for part
      \ref{partial} and \ref{total} would be identical.

      \label{partial}

    }

  \item $\frac{df}{dx}$ where $f(x,y) = x^2 + y^2$ and $y(x) = 1/x$

    \boxedanswer{
      \begin{align*}
        \frac{df}{dx} &= \frac{d}{dx}\left[x^2 + \frac{1}{x^2}\right] \\
        \Aboxed{\frac{df}{dx} &= 2x - \frac{2}{x^3}}
      \end{align*}
    }

    \label{total}
\end{enumerate}

\section*{Problem 6}

Find the critical points $(x, y)$ of the function $f (x, y) =
\ln(xy)$ subject to the constraint $2x^2 + y^2 = 1$.
Hint: use Lagrange multipliers.

\boxedanswer{
  \begin{align*}
    f(x,y) &= \ln(xy) \\
    g(x,y) &= 2x^2 + y^2 - 1 = 0 \\
    \mathcal{L}(x,y\lambda) &= f(x,y) + \lambda \cdot  g(x,y) \\
    \mathcal{L}(x,y\lambda) &= \ln(xy) + \lambda (2x^2 + y^2 - 1) \\
    \nabla\mathcal{L}(x,y\lambda) &= \left(\frac{1}{x} + \lambda 4x,
    \frac{1}{y} + \lambda2y, 2x^2 + y^2 - 1\right) \\
    &
    \begin{cases}
      \frac{1}{x} + \lambda 4x &= 0 \\
      \frac{1}{y} + \lambda 2x &= 0 \\
      2x^2 + y^2 - 1 &= 0 \\
    \end{cases} \\
    \frac{1}{x} + \lambda 4x &= 0 \\
    \frac{1}{y} + \lambda 2y &= 0 \\
  \end{align*}
}

\boxedanswer{
  \begin{align*}
    x &= -\frac{1}{4\lambda} \\
    y &= -\frac{1}{2\lambda} \\
    0 &= 2\left(-\frac{1}{4\lambda} \right)^2 +
    \left(-\frac{1}{2\lambda} \right)^2 - 1 \\
    1 &= \frac{1}{8\lambda^2} + \frac{1}{4\lambda^2} \\
    \lambda &= \pm\sqrt\frac{3}{8} \\
    \frac{1}{x} \pm \sqrt\frac{3}{8} 4x &= 0 \\
    \mp\sqrt\frac{3}{8} x^2 &= 1 \\
    x &= \sqrt{\mp\frac{1}{4}\sqrt{\frac{8}{3}}} \\
    \Aboxed{x &= \frac{1}{2}\sqrt{\mp\sqrt{\frac{8}{3}}}} \\
    \frac{1}{y} \pm \sqrt\frac{3}{8} 2y &= 0 \\
    \Aboxed{y &= \sqrt{\mp \frac{1}{2}\sqrt\frac{8}{3}}}
  \end{align*}

}

\section*{Problem 7}

Calculate the following integrals:

\begin{enumerate}[a.]
  \item $\int_0^\infty dxe^{-\lambda x}$

    \boxedanswer{
      \begin{align*}
        \int_0^\infty dx e^{-\lambda x} &= -\frac{1}{\lambda}
        \left[e^{-\lambda x}\right]_0^\infty \\
        \Aboxed{\int_0^\infty dx e^{-\lambda x} &= \frac{1}{\lambda}}
      \end{align*}
    }

  \item $\int_a^b dt \frac{1}{t} $

    \boxedanswer{
      \begin{align*}
        \int_a^b dt \frac{1}{t} &= \ln[t]_a^b \\
        \Aboxed{\int_a^b dt \frac{1}{t} &= \ln\left[\frac{b}{a}\right]}
      \end{align*}
    }

  \item $\int_{-\infty}^\infty d\zeta e^{-\frac{\zeta^2}{2\sigma^2}}$

    \boxedanswer{

      This is the error function in another form. It has no
      analytical solution. But at $-\infty$ it is $-1$, and it is $1$
      at $\infty$. We can use this to evalute the definite integral.

      \begin{align*}
        \int_{-\infty}^{\infty} d\zeta e^{-\frac{\zeta^2}{2\sigma^2}}
        &=
        \left[\frac{\sqrt{2\sigma^2}\sqrt\pi}{2}\operatorname{erf}\left(\frac{x}{\sqrt{2\sigma^2}}\right)\right]_{-\infty}^\infty
        \\
        \int_{-\infty}^{\infty} d\zeta e^{-\frac{\zeta^2}{2\sigma^2}}
        &= \left[\frac{\sqrt{2\sigma^2}\sqrt\pi}{2} +
        \frac{\sqrt{2\sigma^2}\sqrt\pi}{2}\right] \\
        \Aboxed{\int_{-\infty}^{\infty} d\zeta e^{-\frac{\zeta^2}{2\sigma^2}}
        &= \sqrt{2\sigma^2\pi}}
      \end{align*}
    }

\end{enumerate}

\section*{Problem 8}

Consider the function $F(x, y) = \cos(x)\sin(y)$

\begin{enumerate}[a.]
  \item Find a power series expansion of $F (x, y)$ to quadratic order
    in $\delta x = x − x_0$ and $\delta y = y − y_0$
    about $(x_0, y_0) = (\pi, \pi/2)$. Does this point describe a
    maximum, minimum, or saddle?

    \boxedanswer{
      \begin{align*}
        F(x,y) &\approx F_0 + F_x\delta x + F_y\delta y +
        \frac{1}{2}(F_{xx}\delta x^2 + 2F_{xy}\delta x\delta y +
        F_{yy}\delta y^2) \\
        F(x,y) &\approx  \cos(\pi)\sin(\pi/2) -
        \sin(\pi)\sin(\pi/2)(x - \pi) + \cos(\pi)\cos(\pi/2)(y - \pi/2) + \\
        & \frac{1}{2}\left(-\cos(\pi)\sin(\pi/2)(x - \pi)^2 -
          2\sin(\pi)\cos(\pi/2)(x - \pi)(y - \pi/2) -
        \cos(\pi)\sin(\pi/2)(y - \pi/2)^2\right) \\
        \Aboxed{F(x,y) &\approx  1 - \frac{(y - \pi/2)^2}{2}}
      \end{align*}
      This is a minimum, as $\cos(\pi) = -1$ is a mimimum and
      $\sin(\pi/2) = 1$ is a maximum, This makes the function's
      smallest possible value $-1$.
    }

  \item Same question as above, except at $(x_0, y_0) = (0, \pi/2)$.
    \boxedanswer{

      \begin{align*}
        F(x,y) &\approx  \cos(0)\sin(\pi/2) -
        \sin(0)\sin(\pi/2)(x) + \cos(0)\cos(\pi/2)(y - \pi/2) + \\
        & \frac{1}{2}\left(-\cos(0)\sin(\pi/2)(x)^2 -
          2\sin(0)\cos(\pi/2)(x)(y - \pi/2) +
        \cos(0)\sin(\pi/2)(y - \pi/2)^2\right) \\
        \Aboxed{F(x,y) &\approx 1 + \frac{(y - \pi/2)^2}{2}}
      \end{align*}

      This is a maximum, as both $\cos(x) = \sin(y) = 1$.
    }

    \pagebreak

  \item Similarly, at $(x_0, y_0) = (\pi/2, \pi)$
    \boxedanswer{
      \begin{align*}
        F(x,y) &\approx F_0 + F_x\delta x + F_y\delta y +
        \frac{1}{2}(F_{xx}\delta x^2 + 2F_{xy}\delta x\delta y +
        F_{yy}\delta y^2) \\
        F(x,y) &\approx  \cos(\pi/2)\sin(\pi) -
        \sin(\pi/2)\sin(\pi)(x - \pi/2) + \cos(\pi/2)\cos(\pi)(y - \pi) + \\
        & \frac{1}{2}\left(-\cos(\pi/2)\sin(\pi)(x - \pi/2)^2 -
          2\sin(\pi/2)\cos(\pi)(x - \pi/2)(y - \pi) -
        \cos(\pi/2)\sin(\pi)(y - \pi)^2\right) \\
        \Aboxed{F(x,y) &\approx  1 - (x - \pi/2)(y - \pi)}
      \end{align*}
      This is a saddle point, as both $\cos(\pi/2) = \sin(\pi) = 0$.
    }
\end{enumerate}

\end{document}
