\documentclass[10pt]{exam}
\usepackage{amsthm}
\usepackage{libertine}
\usepackage[utf8]{inputenc}
\usepackage[T1]{fontenc}
\usepackage[margin=0.25in, includefoot]{geometry}
\usepackage{amsmath,amssymb}
\usepackage{multicol}
\usepackage[shortlabels]{enumitem}
\usepackage{cancel}
\usepackage{listings}
\usepackage{tikz}
\usepackage{float}
\usepackage{mathtools}
\usepackage{wrapfig, lipsum}
\usepackage{chemformula}
\usepackage{empheq}
\usepackage[hypcap=false]{caption}
\usepackage{subcaption}
\usepackage[colorlinks=true, linkcolor=blue, urlcolor=blue]{hyperref}
% \usepackage[normalem]{ulem}
\usepackage{graphicx}
\usepackage{bm}
\usepackage{accents}
\usepackage[T1]{fontenc}
\usepackage[font=small,labelfont=bf,tableposition=top]{caption}
\usepackage{siunitx}
\usepackage{minted}
\usepackage{etoolbox}
\usepackage{multirow}
\usepackage[version=4]{mhchem}
\usepackage{pgfplots}
\usepackage{svg}
\usepackage{microtype}
\usepackage{blindtext}
\usepackage{booktabs}
\usepackage{booktabs}
\usepackage{pdfpages}
\usepackage{nicematrix}

\pgfplotsset{compat=1.18}


\AtBeginEnvironment{minted}{
\fontsize{8}{10}\selectfont}
\newcommand{\fahrenheit}{^\circ{F}}
\newcommand*\widefbox[1]{\fbox{\hspace{2em}#1\hspace{2em}}}
\newcommand{\msout}[1]{
    \text{\sout{$#1$}}
}
\usetikzlibrary{shapes}
\DeclareSIUnit\year{y}
\sisetup{
group-digits=integer,
group-minimum-digits=3,
group-separator={,}
}
% \newcommand{\class}{NUEN 604 - 600}
% \newcommand{\examnum}{Homework 1}
% \newcommand{\examdate}{September 11\textsuperscript{th}}
% \newcommand{\timelimit}{}
% \newcommand{\laplace}{\mathcal{L}}

\newcommand{\boxedanswer}[3][0.93]{
    \fbox{
        \begin{minipage}{#1\textwidth}
            #2
        \end{minipage}
        }
    }



% Additional SI unit for Fahrenheit
\DeclareSIUnit\fahrenheit{\degree F}

\begin{document}

% Include title page
\begin{titlepage}
    \begin{center}
        \vspace*{1cm}
            
        \Huge
        \textbf{Homework \#1}
            
        \vspace{0.5cm}
        \LARGE
        MSEN 640 - 600
            
        \vspace{1.5cm}
        
        By: \\

        \textbf{Nathaniel Thomas}

        On Behalf of:

        \textbf{Dr. Michael Dimistriev}

        \vfill

        \vspace{0.8cm}
            
        \includesvg[width=0.4\textwidth]{"./assets/a&m_logo.svg"}
            
        \Large
        Nuclear Engineering\\
        Texas A\&M University\\
        September 12\textsuperscript{th}, 2025
            
    \end{center}
\end{titlepage}


\pagebreak

\section*{Problem 2.4.1}

Which of the following differential equations is linear, in the sense
that, if some function $\psi(z)$ is a
solution (and this may well be a different function for each equation
below), so also is the function
$\phi(z) = a\psi(z)$, where $a$ is an arbitrary constant? Justify your answers.

\begin{enumerate}[(i)]
  \item $z\frac{d\psi(z)}{dz} + g(z)\psi(z) = 0$ where $g(z)$ is some
    specific function

    \boxedanswersmall{
      Some answer:
      \begin{align*}
        1 + 1 = 2
      \end{align*}
    }

  \item $\psi(z)\frac{d\psi(z)}{dz} + \psi(z) = 0$

    \boxedanswersmall{
      Some answer:
      \begin{align*}
        1 + 1 = 2
      \end{align*}
    }

\end{enumerate}

\pagebreak

\section*{Problem 2.6.1}

An electron is in a potential well of thickness \SI{1}{\nano\meter},
with infinitely high potential barriers on either
side. It is in the lowest possible energy state in this well. What
would be the probability of finding the
electron between \SI{0.1}{\nano\meter} and \SI{0.2}{\nano\meter} from
one side of the well?

\boxedanswer{
  Some answer:
  \begin{align*}
    1 + 1 = 2
  \end{align*}
}

\pagebreak

\section*{Problem 2.6.2}

Which of the following functions have a definite parity relative to
the point $x = 0$ (i.e., we are
interested in their symmetry relative to $x = 0$)? For those that
have a definite parity, state whether it
is even or odd.

\begin{enumerate}[(i)]
  \item $\sin(x)$

    \boxedanswersmall{
      Some answer:
      \begin{align*}
        1 + 1 = 2
      \end{align*}
    }

  \item $\exp(ix)$

    \boxedanswersmall{
      Some answer:
      \begin{align*}
        1 + 1 = 2
      \end{align*}
    }

\end{enumerate}

\pagebreak

\section*{Problem 2.6.3}

Consider the problem of an electron in a one-dimensional "infinite"
potential well of width $L_z$ in
the $z$ direction (i.e., the potential energy is infinite for $z < 0$ and
  for $z > L_z$, and, for simplicity, zero
for other values of $z$). For each of the following functions, in
exactly the form stated, is this function
a solution of the time-independent Schrödinger equation?

\begin{enumerate}[(i)]
  \item $\sin(7\pi z / L_z)$

    \boxedanswersmall{
      Some answer:
      \begin{align*}
        1 + 1 = 2
      \end{align*}
    }

  \item $\cos(2\pi z / L_z)$

    \boxedanswersmall{
      Some answer:
      \begin{align*}
        1 + 1 = 2
      \end{align*}
    }

\end{enumerate}

\pagebreak

\section*{Problem 2.7.1}

Which of the following pairs of functions are orthogonal on the
interval $-1$ to $+1$?

\begin{enumerate}
  \item[(i)] $x, x^2$

    \boxedanswersmall{
      Some answer:
      \begin{align*}
        1 + 1 = 2
      \end{align*}
    }

  \item[(v)] $\exp(-2\pi ix), \exp(2\pi ix)$

    \boxedanswersmall{
      Some answer:
      \begin{align*}
        1 + 1 = 2
      \end{align*}
    }

\end{enumerate}

\pagebreak

\section*{Problem 2.8.2}

An electron wave of energy \SI{0.5}{\electronvolt} is incident on an
infinitely thick potential barrier of height
\SI{1}{\electronvolt}. Is the electron more likely to be found (a) within the
first \SI{1}{\angstrom} of the barrier, or (b)
somewhere further into the barrier?

\boxedanswer{
  Some answer:
  \begin{align*}
    1 + 1 = 2
  \end{align*}
}

\end{document}
